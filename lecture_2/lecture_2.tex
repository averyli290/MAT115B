% Adapted from the MIT/Berkeley scribe note template

\documentclass[11pt,letter]{article}

\usepackage{amsmath}
\usepackage{amsthm}
\usepackage{amssymb}
\usepackage{verbatim}
\usepackage{dsfont}
\usepackage{bm}
\usepackage{algorithm}

\usepackage{xcolor}
\definecolor{green}{rgb}{0.0, 0.5, 0.0}
\usepackage[colorlinks,citecolor=blue,linkcolor=magenta,bookmarks=true,hypertexnames=false]{hyperref}
\usepackage[nameinlink,capitalize]{cleveref}
\providecommand\algorithmname{algorithm}
\crefname{equation}{equation}{equations}
\crefformat{equation}{(#2#1#3)}
\crefname{lemma}{lemma}{lemmas}
\crefname{claim}{claim}{claims}
\crefname{theorem}{theorem}{theorems}
\crefname{proposition}{proposition}{propositions}
\crefname{corollary}{corollary}{corollaries}
\crefname{claim}{claim}{claims}
\crefname{remark}{remark}{remarks}
\crefname{definition}{definition}{definitions}
\crefname{fact}{fact}{facts}
\crefname{question}{question}{questions}
\crefname{condition}{condition}{conditions}
\crefname{algorithm}{algorithm}{algorithms}
\crefname{assumption}{assumption}{assumptions}
\crefname{notation}{notation}{notation}
\crefname{cond}{Condition}{Conditions}
\crefname{problem}{problem}{problems}
\crefname{formula}{formula}{formulas}

% Probability
\renewcommand{\Pr}{\operatorname*{\mathbb{P}}}
\newcommand{\Var}{\operatorname*{\mathrm{Var}}}
\newcommand{\Cov}{\operatorname*{\mathrm{Cov}}}
\newcommand{\Exp}{\operatorname*{\mathbb{E}}}
\newcommand{\from}{\leftarrow}
\newcommand{\negl}{\mathrm{negl}}

% Asymptotics
\newcommand{\poly}{\operatorname*{\mathrm{poly}}}
\newcommand{\polylog}{\operatorname*{\mathrm{polylog}}}

% Distributions
\newcommand{\Normal}{\mathcal{N}}
\newcommand{\Bin}{\mathrm{Bin}}
\newcommand{\Poi}{\mathrm{Poi}}
\newcommand{\Unif}{\mathrm{Unif}}
\newcommand{\Bernoulli}{\mathrm{Ber}}
\newcommand{\Geom}{\mathrm{Geom}}
\newcommand{\DistD}{\mathcal{D}}
\newcommand{\DistG}{\mathcal{G}}
\newcommand{\DTV}{d_\mathrm{TV}}
\newcommand{\DKL}{D_\mathrm{KL}}
\renewcommand{\DH}{d_\mathrm{H}}
\newcommand{\DK}{d_\mathrm{K}}

% Sets and other mathematical constructs
\renewcommand{\vec}[1]{\boldsymbol{\mathbf{#1}}}
\newcommand{\PSet}{\mathcal{P}}
\newcommand{\Real}{\mathbb{R}}
\newcommand{\Rational}{\mathbb{Q}}
\newcommand{\Nat}{\mathbb{N}}
\newcommand{\Int}{\mathbb{Z}}
\newcommand{\argmax}{\operatorname*{\mathrm{arg\,max}}}
\newcommand{\argmin}{\operatorname*{\mathrm{arg\,min}}}
\renewcommand{\implies}{\Rightarrow}
\newcommand{\2}{\{0, 1\}}
\newcommand{\1}{\mathds{1}}

% Calculus/Analysis
\renewcommand{\d}{\mathrm{d}}
\newcommand{\Diff}[2][]{\frac{\d#1}{\d#2}}
\newcommand{\Grad}{\nabla}
\newcommand{\Del}[2][]{\frac{\partial#1}{\partial#2}}

% Linear algebra
\newcommand{\tr}{\mathrm{tr}}
\newcommand{\lmin}{\lambda_{\min}}
\newcommand{\lmax}{\lambda_{\max}}

% Miscellaneous
\newcommand{\eps}{\epsilon}
\newcommand{\cA}{\mathcal{A}}
\newcommand{\cB}{\mathcal{B}}
\newcommand{\cC}{\mathcal{C}}
\newcommand{\cD}{\mathcal{D}}
\newcommand{\cE}{\mathcal{E}}
\newcommand{\cF}{\mathcal{F}}
\newcommand{\cG}{\mathcal{G}}
\newcommand{\cH}{\mathcal{H}}
\newcommand{\cI}{\mathcal{I}}
\newcommand{\cJ}{\mathcal{J}}
\newcommand{\cK}{\mathcal{K}}
\newcommand{\cL}{\mathcal{L}}
\newcommand{\cM}{\mathcal{M}}
\newcommand{\cN}{\mathcal{N}}
\newcommand{\cO}{\mathcal{O}}
\newcommand{\cP}{\mathcal{P}}
\newcommand{\cQ}{\mathcal{Q}}
\newcommand{\cR}{\mathcal{R}}
\newcommand{\cS}{\mathcal{S}}
\newcommand{\cT}{\mathcal{T}}
\newcommand{\cU}{\mathcal{U}}
\newcommand{\cV}{\mathcal{V}}
\newcommand{\cW}{\mathcal{W}}
\newcommand{\cX}{\mathcal{X}}
\newcommand{\cY}{\mathcal{Y}}
\newcommand{\cZ}{\mathcal{Z}}
\newcommand{\Chi}{\cX}
\newcommand{\sgn}{\mathrm{sgn}}

\newcommand{\new}[1]{{\color{red} #1}}

% Project specific commands
%

\newcounter{nLectures}
\newcounter{nTheorems}[nLectures]

\newtheorem{theorem}[nTheorems]{Theorem}
\newtheorem{corollary}[nTheorems]{Corollary}
\newtheorem{conjecture}[nTheorems]{Conjecture}
\newtheorem{lemma}[nTheorems]{Lemma}
\newtheorem{proposition}[nTheorems]{Proposition}
\newtheorem{protocol}[nTheorems]{Protocol}
\newtheorem{claim}[nTheorems]{Claim}
\newtheorem{fact}[nTheorems]{Fact}

\theoremstyle{definition}
\newtheorem{definition}[nTheorems]{Definition}
\newtheorem{problem}[nTheorems]{Problem}
\newtheorem{intuition}[nTheorems]{Intuition}
\newtheorem{idea}[nTheorems]{Idea}
\newtheorem{exercise}[nTheorems]{Exercise}
\newtheorem{remark}[nTheorems]{Remark}
\newtheorem{formula}[nTheorems]{Formula}

\makeatletter
	\let\c@algorithm\c@nTheorems
\makeatother

\usepackage{xpatch}
\makeatletter
\AtBeginDocument{\xpatchcmd{\@thm}{\thm@headpunct{.}}{\thm@headpunct{}}{}{}}
\makeatother

\hbadness=10000
\vbadness=10000

\setlength{\oddsidemargin}{.25in}
\setlength{\evensidemargin}{.25in}
\setlength{\textwidth}{6in}
\setlength{\topmargin}{-0.4in}
\setlength{\textheight}{9in}

\newcommand{\lecture}[4]{
	\setcounter{nLectures}{#1}
	\renewcommand{\thenTheorems}{\arabic{nLectures}.\arabic{nTheorems}}
	\renewcommand{\thealgorithm}{\arabic{nLectures}.\arabic{nTheorems}}
	\noindent
	\begin{center}
	\framebox{
		\vbox{
		\hbox to 5.78in { {\bf MAT 115B Number Theory
		\hfill Winter 2025} }
		\hbox to 5.78in { {\it \hfill #4} }
		\vspace{4mm}
		\hbox to 5.78in { {\Large \hfill Lecture \arabic{nLectures} \hfill} }
		\vspace{2mm}
		\hbox to 5.78in { {\it Lecturer: Elena Fuchs \hfill Scribe: #3} }
		}
	}
	\end{center}
	\vspace*{4mm}
}

\begin{document}
\lecture{2}{}{Avery Li}{January 8, 2025}

Each block has $n$ different residues mod $n$ in $\{0,\dots,n-1\}$.

\begin{fact}
If $y\equiv x\mod n$, then $\gcd(n,x)=\gcd(n,y)$. Then, the number of prime numbers in each
block relatively prime to n is $\varphi(n)$.
\end{fact}

There are $\varphi(n)$ relatively prime numbers in each block and $\varphi(m)$ blocks, so
$\varphi(mn)=\varphi(m)\varphi(n)$.

\begin{exercise}
	$\varphi(p^\ell)=p^\ell-p^{\ell-1}$. Then, if $n=p_1^{r_1}\cdots p_1^{r_1}$, $\varphi(n)=\prod(p_i^{r_i}-p_i^{r_i-1})
\implies\varphi(n)=n\cdot\prod(1-\frac{1}{p_i})$.
\end{exercise}

\begin{theorem}\label{thm:thm_2_3}
	Suppose $f$ is a multiplicative arithmetic function, define $$F(x)=\sum_{d\vert x,d\geq 1}f(x),$$ then $F(x)$ is multiplciative.
\end{theorem}

\begin{corollary}
	$\nu(x)$ is multiplicative.
\end{corollary}
\begin{proof}
	$\nu(x)=\sum_{d\vert x,d\geq 1}1$, $f(x)=1$ is trivially multiplicative, so by \hyperref[thm:thm_2_3]{2.3}, $\nu(x)$ is multiplicative.
\end{proof}

\begin{corollary}
	$\sigma(x)$ is multiplicative.
\end{corollary}
\begin{proof}
	\textit{To be written.}
\end{proof}

\begin{proof}[Proof of Theorem 2.3]
	Let $(m, n)=1$, note that for all $d\vert mn$, there exists $d_1\vert m$, $d_2\vert n$ where $d_1d_2=d$ and $(d_1, d_2)=1$.
	Then, we have that
	\begin{align*}
		F(mn)&=\sum_{d\vert mn}f(d) \\
		&=\sum_{d_1\vert m}\sum_{d_2\vert n}f(d_1d_2)\\
		&=\sum_{d_1\vert m}\sum_{d_2\vert n}f(d_1)f(d_2)\\
		&=\sum_{d_1\vert m}f(d_1)\sum_{d_2\vert n}f(d_2)\\
		&=F(m)F(n).\\
	\end{align*}
\end{proof}

\begin{formula}
	$\nu(p^r)=r+1\implies$ if $n=p_1^{r_1}\cdots p_k^{r_k},\nu(n)=\prod(r_i+1)$.
\end{formula}

\begin{remark}\label{rem:rem_2_7}
	$(y^n-1)=(y-1)(y^{n-1}+\cdots+y+1)$.
\end{remark}

\begin{formula}
	$\sigma(p^r)=1+p+\cdots+p^r=\frac{p^{r_i+1}-1}{p-1}\implies$ if
	$n=p_1^{r_1}\cdots p_k^{r_k},\sigma(n)=\prod\frac{p_i^{r_i}-1}{p_i-1}$.
\end{formula}

\begin{definition}
	$n\geq 1$ is perfect if $\sigma(n)=2n$.
\end{definition}

Some examples are as follows: $6=1+2+3, 28=1+2+4+7+14, 496=\sum_{d\vert 496}d$. Note that we can
rewrite each of these using the form $n=2^{p-1}(2^p-1)$ where $p$ and $2^p-1$ are prime,
using $p=2,3,5$ respectively.

\begin{remark}
	$2^p-1$ is prime $\implies p$ is prime.
\end{remark}

\begin{proof}
	Assume towards a contradiction that $p$ is not prime, then $p=ab$ where $a,b>1$.
	By \hyperref[rem:rem_2_7]{2.7}, $2^{ab}-1=(2^a-1)(2^{a(b-1)}+\cdots+1)$, which is a contradiction.
\end{proof}

\begin{theorem}
	A number of the form $n=2^{p-1}(2^p-1)$ is perfect if $2^p-1$ is prime.
\end{theorem}
\begin{proof}
	\begin{align*}
		\sigma(n)&=\sigma(2^{p-1}(2^p-1)) \\
		&=\sigma(2^{p-1})\sigma(2^p-1) \\
		&=\frac{2^p-1}{2-1}(2^p-1 + 1)\\
		&=(2^p-1)(2^p)\\
		&=2\cdot2^{p-1}(2^p-1)\\
		&=2n.
	\end{align*}
\end{proof}

\end{document}