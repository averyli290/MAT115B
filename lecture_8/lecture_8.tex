% Adapted from the MIT/Berkeley scribe note template

\documentclass[11pt,letter]{article}

\usepackage{amsmath}
\usepackage{amsthm}
\usepackage{amssymb}
\usepackage{verbatim}
\usepackage{dsfont}
\usepackage{bm}
\usepackage{algorithm}

\usepackage{xcolor}
\definecolor{green}{rgb}{0.0, 0.5, 0.0}
\usepackage[colorlinks,citecolor=blue,linkcolor=magenta,bookmarks=true,hypertexnames=false]{hyperref}
\usepackage[nameinlink,capitalize]{cleveref}
\providecommand\algorithmname{algorithm}
\crefname{equation}{equation}{equations}
\crefformat{equation}{(#2#1#3)}
\crefname{lemma}{lemma}{lemmas}
\crefname{claim}{claim}{claims}
\crefname{theorem}{theorem}{theorems}
\crefname{proposition}{proposition}{propositions}
\crefname{corollary}{corollary}{corollaries}
\crefname{claim}{claim}{claims}
\crefname{remark}{remark}{remarks}
\crefname{definition}{definition}{definitions}
\crefname{fact}{fact}{facts}
\crefname{question}{question}{questions}
\crefname{condition}{condition}{conditions}
\crefname{algorithm}{algorithm}{algorithms}
\crefname{assumption}{assumption}{assumptions}
\crefname{notation}{notation}{notation}
\crefname{cond}{Condition}{Conditions}
\crefname{problem}{problem}{problems}
\crefname{formula}{formula}{formulas}
\crefname{example}{example}{examples}

% Number Theory
\newcommand{\legendre}[2]{\ensuremath{\left( \frac{#1}{#2} \right) }}
\newcommand{\Mod}[1]{\ (\mathrm{mod}\ #1)}

% Probability
\renewcommand{\Pr}{\operatorname*{\mathbb{P}}}
\newcommand{\Var}{\operatorname*{\mathrm{Var}}}
\newcommand{\Cov}{\operatorname*{\mathrm{Cov}}}
\newcommand{\Exp}{\operatorname*{\mathbb{E}}}
\newcommand{\from}{\leftarrow}
\newcommand{\negl}{\mathrm{negl}}

% Asymptotics
\newcommand{\poly}{\operatorname*{\mathrm{poly}}}
\newcommand{\polylog}{\operatorname*{\mathrm{polylog}}}

% Distributions
\newcommand{\Normal}{\mathcal{N}}
\newcommand{\Bin}{\mathrm{Bin}}
\newcommand{\Poi}{\mathrm{Poi}}
\newcommand{\Unif}{\mathrm{Unif}}
\newcommand{\Bernoulli}{\mathrm{Ber}}
\newcommand{\Geom}{\mathrm{Geom}}
\newcommand{\DistD}{\mathcal{D}}
\newcommand{\DistG}{\mathcal{G}}
\newcommand{\DTV}{d_\mathrm{TV}}
\newcommand{\DKL}{D_\mathrm{KL}}
\renewcommand{\DH}{d_\mathrm{H}}
\newcommand{\DK}{d_\mathrm{K}}

% Sets and other mathematical constructs
\renewcommand{\vec}[1]{\boldsymbol{\mathbf{#1}}}
\newcommand{\PSet}{\mathcal{P}}
\newcommand{\Real}{\mathbb{R}}
\newcommand{\Rational}{\mathbb{Q}}
\newcommand{\Nat}{\mathbb{N}}
\newcommand{\Int}{\mathbb{Z}}
\newcommand{\argmax}{\operatorname*{\mathrm{arg\,max}}}
\newcommand{\argmin}{\operatorname*{\mathrm{arg\,min}}}
\renewcommand{\implies}{\Rightarrow}
\newcommand{\2}{\{0, 1\}}
\newcommand{\1}{\mathds{1}}

% Calculus/Analysis
\renewcommand{\d}{\mathrm{d}}
\newcommand{\Diff}[2][]{\frac{\d#1}{\d#2}}
\newcommand{\Grad}{\nabla}
\newcommand{\Del}[2][]{\frac{\partial#1}{\partial#2}}

% Linear algebra
\newcommand{\tr}{\mathrm{tr}}
\newcommand{\lmin}{\lambda_{\min}}
\newcommand{\lmax}{\lambda_{\max}}

% Miscellaneous
\newcommand{\eps}{\epsilon}
\newcommand{\cA}{\mathcal{A}}
\newcommand{\cB}{\mathcal{B}}
\newcommand{\cC}{\mathcal{C}}
\newcommand{\cD}{\mathcal{D}}
\newcommand{\cE}{\mathcal{E}}
\newcommand{\cF}{\mathcal{F}}
\newcommand{\cG}{\mathcal{G}}
\newcommand{\cH}{\mathcal{H}}
\newcommand{\cI}{\mathcal{I}}
\newcommand{\cJ}{\mathcal{J}}
\newcommand{\cK}{\mathcal{K}}
\newcommand{\cL}{\mathcal{L}}
\newcommand{\cM}{\mathcal{M}}
\newcommand{\cN}{\mathcal{N}}
\newcommand{\cO}{\mathcal{O}}
\newcommand{\cP}{\mathcal{P}}
\newcommand{\cQ}{\mathcal{Q}}
\newcommand{\cR}{\mathcal{R}}
\newcommand{\cS}{\mathcal{S}}
\newcommand{\cT}{\mathcal{T}}
\newcommand{\cU}{\mathcal{U}}
\newcommand{\cV}{\mathcal{V}}
\newcommand{\cW}{\mathcal{W}}
\newcommand{\cX}{\mathcal{X}}
\newcommand{\cY}{\mathcal{Y}}
\newcommand{\cZ}{\mathcal{Z}}
\newcommand{\Chi}{\cX}
\newcommand{\sgn}{\mathrm{sgn}}

\newcommand{\new}[1]{{\color{red} #1}}


% Project specific commands
%

\newcounter{nLectures}
\newcounter{nTheorems}[nLectures]

\newtheorem{theorem}[nTheorems]{Theorem}
\newtheorem{corollary}[nTheorems]{Corollary}
\newtheorem{conjecture}[nTheorems]{Conjecture}
\newtheorem{lemma}[nTheorems]{Lemma}
\newtheorem{proposition}[nTheorems]{Proposition}
\newtheorem{protocol}[nTheorems]{Protocol}
\newtheorem{claim}[nTheorems]{Claim}
\newtheorem{fact}[nTheorems]{Fact}

\theoremstyle{definition}
\newtheorem{definition}[nTheorems]{Definition}
\newtheorem{problem}[nTheorems]{Problem}
\newtheorem{intuition}[nTheorems]{Intuition}
\newtheorem{idea}[nTheorems]{Idea}
\newtheorem{exercise}[nTheorems]{Exercise}
\newtheorem{remark}[nTheorems]{Remark}
\newtheorem{formula}[nTheorems]{Formula}
\newtheorem{example}[nTheorems]{Example}

\makeatletter
	\let\c@algorithm\c@nTheorems
\makeatother

\usepackage{xpatch}
\makeatletter
\AtBeginDocument{\xpatchcmd{\@thm}{\thm@headpunct{.}}{\thm@headpunct{}}{}{}}
\makeatother

\hbadness=10000
\vbadness=10000

\setlength{\oddsidemargin}{.25in}
\setlength{\evensidemargin}{.25in}
\setlength{\textwidth}{6in}
\setlength{\topmargin}{-0.4in}
\setlength{\textheight}{9in}

\newcommand{\lecture}[4]{
	\setcounter{nLectures}{#1}
	\renewcommand{\thenTheorems}{\arabic{nLectures}.\arabic{nTheorems}}
	\renewcommand{\thealgorithm}{\arabic{nLectures}.\arabic{nTheorems}}
	\noindent
	\begin{center}
	\framebox{
		\vbox{
		\hbox to 5.78in { {\bf MAT 115B Number Theory
		\hfill Winter 2025} }
		\hbox to 5.78in { {\it \hfill #4} }
		\vspace{4mm}
		\hbox to 5.78in { {\Large \hfill Lecture \arabic{nLectures} \hfill} }
		\vspace{2mm}
		\hbox to 5.78in { {\it Lecturer: Elena Fuchs \hfill Scribe: #3} }
		}
	}
	\end{center}
	\vspace*{4mm}
}

\begin{document}
\lecture{8}{}{Avery Li}{January 24, 2025}

\section*{Lecture 7 Recap}
\begin{theorem}\label{thm:thm_8_1}
	Let $p$ be an odd prime, then $\left(\frac{2}{p}\right)=(-1)^{\frac{p^2-1}{8}}$, $2$ is a quadratic residue
	mod $p$ iff $p\equiv 1,7\Mod 8$, and is not when $p\equiv 3, 5\Mod 8$.
\end{theorem}

\section*{Lecture 8}
\begin{example}
    $p=17$, $\left(\frac{2}{17}\right)=1$ because $17\equiv 1\Mod 8$, $6^2\equiv 2\Mod 17$.
\end{example}

\begin{proof}[Proof of Theorem 8.1.]
    We count the values of $2k>\frac{p}{2}$ iff $k>\frac{p}{4}$. There are $n=\frac{p-1}{2}-\lfloor\frac{p}{4}\rfloor$ total $k$.
    Then, Gauss's lemma gives us that $\legendre{2}{p}=(-1)^n=(-1)^{\frac{p-1}{2}-\lfloor\frac{p}{4}\rfloor}$.
    Now, we want to show that $n=\frac{p-1}{2}-\lfloor\frac{p}{4}\rfloor=\frac{p^2-1}{8}\Mod 2$.
    We know that $p$ can be $1,3,5,7\Mod 8$. We will consider $5$, the other cases can be shown similarly.
    Suppose $p=8k+5$ for some $k\in\Int$. Then we have
    \begin{align*}
        LHS &=\frac{8k+5-1}{2}-\lfloor\frac{8k+5}{4}\rfloor \\
            &=4k+2-\lfloor 2k+\frac{5}{4}\rfloor \\
            &=4k+2-2k-1 \\
            &\equiv 1\Mod 2 \\
        RHS&=\frac{(8k+5)^2-1}{8} \\
            &=\frac{64k^2+80k+25-1}{8} \\
            &=8k^2+10k+3 \\
            &\equiv 1\Mod 2\\
        \Longrightarrow LHS&\equiv RHS\Mod 2.
    \end{align*}
\end{proof}

\section*{Quadratic Reciprocity}

\begin{theorem}[Quadratic Reciprocity]\label{thm:thm_8_3}
    Let $p,q$ be distinct odd primes, then $\legendre{p}{q}\cdot\legendre{q}{p}=(-1)^{\frac{p-1}{2}\frac{q-1}{2}}$.
        \[
        =
        \begin{cases}
            1\text{ if }p\equiv 1\Mod 4\text{ or } q\equiv 1\Mod 4 \\
            -1\text{ if }p\equiv q\equiv 3\Mod 4 \\
        \end{cases}
        \]
    Equivalently, $\legendre{q}{p}=(-1)^{\frac{p-1}{2}\cdot\frac{q-1}{2}}\legendre{p}{q}$,
    i.e.
    \[
    \legendre{q}{p}=
    \begin{cases}
        \legendre{p}{q}\text{ if }p\equiv 1\Mod 4\text{ or } q\equiv 1\Mod 4 \\
        -\legendre{p}{q}\text{ otherwise. }\\
    \end{cases}
    \]
\end{theorem}
How is this theorem useful?
\begin{example}
    We can now switch large ``denominators'' to the numerator, and vice versa.
    $\legendre{3}{101}=\legendre{101}{3}=\legendre{2}{3}$ because $101\equiv 1\Mod 4$ and $101\equiv 2\Mod 3$.
\end{example}
\begin{example}
    When is $\legendre{5}{p}=1$? We have that $\legendre{5}{p}=\legendre{p}{5}$ by the theorem. $\legendre{p}{5}=1$
    iff $p\equiv 1,4\Mod 5$ by observation, therefore, $\legendre{5}{p}=1$ iff $p\equiv 1,4\Mod 5$.
\end{example}
\begin{example}
    When is $\legendre{3}{p}=1$ and $\legendre{3}{p}=-1$? We have that $p=2$ works. if $p$ is odd, $p\neq 3$.
    Then, $\legendre{3}{p}=\legendre{p}{3}$ iff $p\equiv 1\Mod 4$. We have that $\legendre{3}{p}=1$ iff $p\equiv 1\Mod 3$,
    so $\legendre{3}{p}=1$ iff $p\equiv 1\Mod{12}$ by Chinese remainder theorem. This can be computed similarly
    for $\legendre{3}{p}=-1$ to get $p\equiv 11\Mod{12}$.
\end{example}
\begin{remark}
    Note that \hyperref[thm:thm_8_3]{8.3} does not work for $p=2$.
\end{remark}

\begin{example}
    \textit{Compute $\legendre{-57}{103}$.}
    
    \begin{align*}
        \legendre{-57}{103}&=\legendre{-1}{103}\legendre{3}{103}\legendre{19}{103} \\
        &=-1\cdot -\legendre{103}{3}\cdot-\legendre{103}{19} &\ref{thm:thm_8_3}\\
        &=-1\cdot\legendre{1}{3}\cdot\legendre{8}{19} &\text{ By LOGIC (using mod)}\\
        &=-1\cdot 1\cdot \legendre{2}{19}^3 &\text{Properties of legendre symbol}\\
        &=-1\cdot \legendre{2}{19}^3 \\
        &=-1\cdot -1  & \ref{thm:thm_8_1} \\
        &=1.\\
    \end{align*}
    
\end{example}

\end{document}