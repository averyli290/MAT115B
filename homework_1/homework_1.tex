\documentclass[11pt]{article}

\usepackage[utf8]{inputenc}
\usepackage{microtype}
\usepackage[margin=1in]{geometry}
\usepackage{enumitem}
\usepackage{amsmath, amssymb, amsthm}
\usepackage{xcolor}
\usepackage{graphicx}
\usepackage{hyperref}

% % COMMANDS
\newcommand{\solution}{{\color{blue} \textsc{\ \ Solution.\ \ }}}


% % SOME ENVIRONMENTS
\newcommand{\exheading}[1]{\section*{Exercise #1}}
\newcounter{exnum}
% \setcounter{exnum}{1} % default 0 start
\newcommand{\exercise}{
    \stepcounter{exnum}
    \exheading{\theexnum}
    }

\newcommand{\bea}{\begin{enumerate}[label={(\alph*)}]}
\newcommand{\ee}{\end{enumerate}}



\title{MAT 115B Homework 1}
\author{Avery Li}
\date{January 15, 2025}

\begin{document}

\maketitle


\begin{flushleft}
1.
\end{flushleft}
Let $n=\prod_{i=1}^k p_i^{r_i}$, then $\phi(n)=\prod_{i=1}^k p_i^{r_i-1}(p_i-1)$. Then we have that
$p^{r_i-1}\vert n-1$ and $p^{r_i-1}\vert n$ for all $i$. Therefore, $r_i=1$ for all i, so $n$ is a product
of distinct primes.

\begin{flushleft}
2.
\begin{align*}
    64&=2^6 \\
    &\implies \sigma(64)=\frac{2^7-1}{2-1}=127\\
    &\implies \tau(64)=6+1=7\\
    105&=3\cdot 5\cdot 7\\
    &\implies \sigma(105)=\frac{3^2-1}{3-1}\cdot\frac{5^2-1}{5-1}\cdot\frac{7^2-1}{7-1}=192\\
    &\implies \tau(105)=2\cdot 2\cdot 2=8\\
    2592&=2^5\cdot 3^4 \\
    &\implies \sigma(2592)=\frac{2^6-1}{2-1}\cdot\frac{3^5-1}{3-1}=7623\\
    &\implies \tau(2592)=6\cdot 5=30\\
    4851&=3^2\cdot 7^2\cdot 11\\
    &\implies \sigma(4851)=\frac{3^3-1}{3-1}\cdot\frac{7^3-1}{7-1}\cdot\frac{11^2-1}{11-1}=8892\\
    &\implies \tau(4851)=(2+1)(2+1)(1+1)=18\\
    111111&=3\cdot 7\cdot 11\cdot 13\cdot 37\\
    &\implies \sigma(111111)=\frac{3^2-1}{3-1}\cdot\frac{7^2-1}{7-1}\cdot\frac{11^2-1}{11-1}\cdot\frac{13^2-1}{13-1}\cdot\frac{37^2-1}{37-1}\\
    &\implies \tau(111111)=(1+1)(1+1)(1+1)(1+1)(1+1)=32\\
    15!&=2^{11}\cdot 3^6\cdot 5^3\cdot 7^2\cdot 11\cdot 13\\
    &\implies \sigma(15!)=\frac{2^{12}-1}{2-1}\cdot\frac{3^7-1}{3-1}\cdot\frac{5^4-1}{5-1}\cdot\frac{7^3-1}{7-1}\cdot\frac{11^2-1}{11-1}\cdot\frac{13^2-1}{13-1}=6686252969760\\
    &\implies \tau(15!)=(11+1)(6+1)(3+1)(2+1)(1+1)(1+1)=4032\\
\end{align*}
\end{flushleft}

\begin{flushleft}
3.
\end{flushleft}
\begin{enumerate}
    \item[a.] $\tau(n)$ is odd when the powers in the prime factorization of $n$ are all even.
    \item[b.] $\sigma(n)$ is odd when $n=x^2$ or $n=2x^2$ for some integer $x\neq 0$. 
\end{enumerate}

\begin{flushleft}
4.
\end{flushleft}
If there exists an $n$ such that $\tau(n)=k$, we can replace the primes in the prime factorization
to obtain another $n'$ where $\tau(n')=k$. We have that $\tau(2^{k-1})=k$, therefore, $\tau(n)=k$ has 
infinitely many solutions because there are infinitely many primes.

\begin{flushleft}
5.
\end{flushleft}
\begin{enumerate}
    \item [a.] There are at most $\sqrt{n}$ divisors $a$ where $a\leq \frac{n}{a}$. Each of these divisors $a$ is paired off with another divisor $\frac{n}{a}$
    where $a\cdot\frac{n}{a}=n$. Therefore, there are at most $2\sqrt{n}$ divisors, so $\tau(n)\leq 2\sqrt{n}$.
    \item [b.] If $a\vert n$ then $2^{a}-1\vert 2^n-1$ because $2^n-1$ can be written $2^{ka}-1$. Therefore, $\tau(n)\leq\tau(2^n-1)$.
\end{enumerate}


\begin{flushleft}
6.
\end{flushleft}
\begin{enumerate}
    \item [a.]
        \[\sum_{d\vert n, d>0}\frac{1}{d}=\sum_{d\vert n, d>0}\frac{\frac{n}{d}}{n}
            =\frac{1}{n}\sum_{d\vert n, d>0}\frac{n}{d}
            =\frac{1}{n}\sigma(n)
            =\frac{\sigma(n)}{n}.\]
    \item [b.]
        \[
            \sum_{d\vert n, d>0}\frac{1}{d}=\sum_{d\vert n, d>0}\frac{\frac{n}{d}}{n}
            =\frac{1}{n}\sum_{d\vert n, d>0}\frac{n}{d}
            =\frac{1}{n}\sigma(n)
            =\frac{\sigma(n)}{n}
            =\frac{2n}{n}
            =2.\]

\end{enumerate}

\begin{flushleft}
7.
\end{flushleft}
\begin{enumerate}
    \item [a.]
        \begin{align*}
            \sigma(16)&=1+2+4+8+16=31\\
            \sigma(31)&=1+31=32\\
        \end{align*}
    \item [b.]
        \begin{align*}
            \sigma(2^{p-1})&=\sum{i=0}^{p-1}2^i=2^p-1\\
            \sigma(2^{p}-1)&=1+2^p-1=2^p\\
        \end{align*}
    \item [c.]
        We have that $\sigma(2^a)=2^{a+1}-1$, then, $\sigma(2^{a+1}-1)=\sum_{d\vert 2^{a+1}-1}$.
        Both $1$ and $2^{a+1}-1$ are divisors, so $\sigma(2^{a+1}-1)\geq 2^{a+1}$. If there are
        any more divisors of $2^{a+1}-1$, then $\sigma(2^{a+1}-1)>2^{a+1}$. Therefore, for
        $2^a$ to be superperfect, $2^{a+1}-1$ must be a Mersenne prime.

\end{enumerate}

\begin{flushleft}
8.
\end{flushleft}
This homework was a 5/10 on difficulty. I spent 4 hours on this homework.

\end{document}
