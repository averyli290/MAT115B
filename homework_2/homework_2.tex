\documentclass[11pt]{article}

\usepackage[utf8]{inputenc}
\usepackage{microtype}
\usepackage[margin=1in]{geometry}
\usepackage{enumitem}
\usepackage{amsmath, amssymb, amsthm}
\usepackage{xcolor}
\usepackage{graphicx}
\usepackage{hyperref}

% % COMMANDS
\newcommand{\solution}{{\color{blue} \textsc{\ \ Solution.\ \ }}}


% % SOME ENVIRONMENTS
\newcommand{\exheading}[1]{\section*{Exercise #1}}
\newcounter{exnum}
% \setcounter{exnum}{1} % default 0 start
\newcommand{\exercise}{
    \stepcounter{exnum}
    \exheading{\theexnum}
    }

\newcommand{\bea}{\begin{enumerate}[label={(\alph*)}]}
\newcommand{\ee}{\end{enumerate}}



\title{MAT 115B Homework 1}
\author{Avery Li}
\date{January 15, 2025}

\begin{document}

\maketitle


\begin{flushleft}
1.
\end{flushleft}
For each distinct prime number dividing n, we can choose to either include or exclude the prime in a product starting with the value $1$.
Doing every combination of this generates every single divisor of $n$ that is square free. There are $2^{\omega(n)}$ combinations,
and $|\mu(d)|=1$ when $d$ is square free. Therefore, $\sum_{d\vert n}|\mu(d)|=2^{\omega(n)}$.


\begin{flushleft}
2.
\end{flushleft}
Let $S=\{p_1,p_2,\dots,p_m\}$
Consider a term in the expanded form of $\prod_i=1^m (1-f(p_i))$. It has the form $f(p_{t_1})f(p_{t_2})\cdots f(p_{t_\ell})(-1)^{\ell}=f(p_{t_1}p_{t_2}\cdots p_{t_\ell})(-1)^\ell$
for some subset of prime divisors $S'=\{p_{t_1},p_{t_2},\dots,p_{t_\ell}\}\subseteq S$. If we take the product over all elements in $S'$
we get $d=p_{t_1}p_{t_2}\cdots p_{t_\ell}$. Over the expanded form of the original product, we choose either $1$ or $-f(p_i)$ for each factor
and this gives every combination of including or excluding each prime, namely, all subsets of the $S$. This gives all divisors $d$ of $n$
which are square free because each prime can only be used once. Finally, $(-1)^\ell=\mu(d)$ because $d$ is prime free for each of the terms
in the product. Therefore, the terms of expanded form of $\prod_{i=1}^m (1-f(p_i))$ are $f(d)\mu(d)$ for each square free $d\vert n$, which is $\sum_{d\vert n}\mu(d)f(d)$.

\begin{flushleft}
3.
\end{flushleft}
By M\"obius inversion, $g(n)=\sum_{d\vert n}\mu(d)f\left(\frac{n}{d}\right)$, so

\begin{align*}
    g(12)&=\mu(1)f(12)+\mu(2)f(6)+\mu(3)f(4)+\mu(4)f(3)+\mu(6)f(2)+\mu(12)f(1)\\
    &= 8+(-1)4+(-1)\frac{8}{3}+0+(1)\frac{4}{3}+0\\
    &=8-4-\frac{4}{3}\\
    &=\frac{8}{3}.
\end{align*}

\begin{flushleft}
4.
\end{flushleft}
Let $n=p_1^{r_1}p_2^{r_2}\cdots p_k^{r_k}$.
We will first prove that $\sum_{d\vert n}\Lambda(d)=\ln n$,

\begin{align*}
    \sum_{d\vert n}\Lambda(d)&=\sum_{i=1}^k\sum_{j=1}^{r_i}\Lambda(p_i^{r_i}) \\
    &=\sum_{i=1}^k\sum_{j=1}^{r_i} \ln p_i \\
    &=\sum_{i=1}^k r_i\ln p_i \\
    &=\sum_{i=1}^k \ln p_i^{r_i} \\
    &=\ln(p_1^{r_1\cdots p_k^{r_k}}) \\
    &=\ln n.
\end{align*}
Now, we will prove that $\Lambda(n)=-\sum_{d\vert n}\mu(d)\ln d$. By M\"obius inversion,
\begin{align*}
    \Lambda(n)&=\sum_{d\vert n}\mu(d)\ln\left(\frac{n}{d}\right) \\
    &=\left(\sum_{d\vert n}\mu(d)\right)\ln n+\sum_{d\vert n}\mu(d)\ln\left(\frac{1}{d}\right) \\
\end{align*}
By a theorem proved in class we can simplify this to 
\begin{align*}
    \left(\sum_{d\vert n}\mu(d)\right)\ln n+\sum_{d\vert n}\mu(d)\ln\left(\frac{1}{d}\right)&=
    0\ln n+\sum_{d\vert n}\mu(d)\ln\left(\frac{1}{d}\right) \\
    \Lambda(n)&=-\sum_{d\vert n}\mu(d)\ln d \\
\end{align*}
Therefore, $\Lambda(n)=-\sum_{d\vert n}\mu(d)\ln d$.

\begin{flushleft}
5.
\end{flushleft}
Let $a,b$ be coprime. By M\"obius inversion, $F$ is multiplicative, and $c\vert a,d\vert b\implies \gcd(c, d)=1$, we have that
\begin{align*}
    f(a)f(b)&=\left(\sum_{c\vert a}\mu(c)F(\frac{a}{c})\right)\left(\sum_{d\vert b}\mu(d)F(\frac{b}{d})\right)\\
    &=\sum_{c\vert a}\sum_{d\vert b}\mu(c)\mu(d)F(\frac{a}{c})F(\frac{b}{d})\\
    &=\sum_{c\vert a}\sum_{d\vert b}\mu(cd)F(\frac{ab}{cd})\\
    &=\sum_{e\vert ab}\mu(e)F(\frac{ab}{e})\\
    &=f(ab).\\
\end{align*}
Therefore, if $F$ is multiplicative, then $f$ is multiplicative.

\begin{flushleft}
6.
\end{flushleft}
This homework was around 4 hours to complete with a 7/10 difficulty level.

\end{document}
