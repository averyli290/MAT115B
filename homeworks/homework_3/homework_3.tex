\documentclass[11pt]{article}

\usepackage[utf8]{inputenc}
\usepackage{microtype}
\usepackage[margin=1in]{geometry}
\usepackage{enumitem}
\usepackage{amsmath, amssymb, amsthm}
\usepackage{xcolor}
\usepackage{graphicx}
\usepackage{hyperref}

% % COMMANDS
\newcommand{\solution}{{\color{blue} \textsc{\ \ Solution.\ \ }}}
\newcommand{\legendre}[2]{\ensuremath{\left( \frac{#1}{#2} \right) }}
\newcommand{\Mod}[1]{\ (\mathrm{mod}\ #1)}
\DeclareMathOperator{\ord}{ord}


% % SOME ENVIRONMENTS
\newcommand{\exheading}[1]{\section*{Exercise #1}}
\newcounter{exnum}
% \setcounter{exnum}{1} % default 0 start
\newcommand{\exercise}{
    \stepcounter{exnum}
    \exheading{\theexnum}
    }

\newcommand{\bea}{\begin{enumerate}[label={(\alph*)}]}
\newcommand{\ee}{\end{enumerate}}



\title{MAT 115B Homework 3}
\author{Avery Li}
\date{January 29, 2025}

\begin{document}

\maketitle


\begin{flushleft}
1.
\end{flushleft}

By observation, we have that the solutions to the $x^2\equiv 1\Mod 8$ are $x\equiv 1, 3, 5, 7\Mod 8$.

\begin{flushleft}
2.
\end{flushleft}

\begin{flushleft}
3.
\end{flushleft}
We have that for $x^2\equiv y^2\Mod{p}\implies x^2-y^2\equiv\Mod{p}\implies (x+y)(x-y)\equiv $.
\begin{align*}
    x^2&\equiv y^2\Mod{p} \\
    x^2-y^2&\equiv 0 \Mod{p} \\
    (x+y)(x-y)&\equiv 0\Mod{p} \\
    \implies x&\equiv \pm y \Mod{p} \\
\end{align*}

\begin{flushleft}
4.

\begin{itemize}
    \item[ a) ]
    \legendre{11}{23}
    \item[ b) ]
    \legendre{-6}{11}
    \item[ c) ]
    \legendre{5}{17}
\end{itemize}

\end{flushleft}

\begin{flushleft}
5.
\end{flushleft}

\begin{flushleft}
6.
\end{flushleft}

\begin{flushleft}
7.
\end{flushleft}

\end{document}
