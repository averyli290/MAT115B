% Adapted from the MIT/Berkeley scribe note template

\documentclass[11pt,letter]{article}

\usepackage{amsmath}
\usepackage{amsthm}
\usepackage{amssymb}
\usepackage{verbatim}
\usepackage{dsfont}
\usepackage{bm}
\usepackage{algorithm}

\usepackage{xcolor}
\definecolor{green}{rgb}{0.0, 0.5, 0.0}
\usepackage[colorlinks,citecolor=blue,linkcolor=magenta,bookmarks=true,hypertexnames=false]{hyperref}
\usepackage[nameinlink,capitalize]{cleveref}
\providecommand\algorithmname{algorithm}
\crefname{equation}{equation}{equations}
\crefformat{equation}{(#2#1#3)}
\crefname{lemma}{lemma}{lemmas}
\crefname{claim}{claim}{claims}
\crefname{theorem}{theorem}{theorems}
\crefname{proposition}{proposition}{propositions}
\crefname{corollary}{corollary}{corollaries}
\crefname{claim}{claim}{claims}
\crefname{remark}{remark}{remarks}
\crefname{definition}{definition}{definitions}
\crefname{fact}{fact}{facts}
\crefname{question}{question}{questions}
\crefname{condition}{condition}{conditions}
\crefname{algorithm}{algorithm}{algorithms}
\crefname{assumption}{assumption}{assumptions}
\crefname{notation}{notation}{notation}
\crefname{cond}{Condition}{Conditions}
\crefname{problem}{problem}{problems}
\crefname{formula}{formula}{formulas}
\crefname{example}{example}{examples}

% Probability
\renewcommand{\Pr}{\operatorname*{\mathbb{P}}}
\newcommand{\Var}{\operatorname*{\mathrm{Var}}}
\newcommand{\Cov}{\operatorname*{\mathrm{Cov}}}
\newcommand{\Exp}{\operatorname*{\mathbb{E}}}
\newcommand{\from}{\leftarrow}
\newcommand{\negl}{\mathrm{negl}}

% Asymptotics
\newcommand{\poly}{\operatorname*{\mathrm{poly}}}
\newcommand{\polylog}{\operatorname*{\mathrm{polylog}}}

% Distributions
\newcommand{\Normal}{\mathcal{N}}
\newcommand{\Bin}{\mathrm{Bin}}
\newcommand{\Poi}{\mathrm{Poi}}
\newcommand{\Unif}{\mathrm{Unif}}
\newcommand{\Bernoulli}{\mathrm{Ber}}
\newcommand{\Geom}{\mathrm{Geom}}
\newcommand{\DistD}{\mathcal{D}}
\newcommand{\DistG}{\mathcal{G}}
\newcommand{\DTV}{d_\mathrm{TV}}
\newcommand{\DKL}{D_\mathrm{KL}}
\renewcommand{\DH}{d_\mathrm{H}}
\newcommand{\DK}{d_\mathrm{K}}

% Sets and other mathematical constructs
\renewcommand{\vec}[1]{\boldsymbol{\mathbf{#1}}}
\newcommand{\PSet}{\mathcal{P}}
\newcommand{\Real}{\mathbb{R}}
\newcommand{\Rational}{\mathbb{Q}}
\newcommand{\Nat}{\mathbb{N}}
\newcommand{\Int}{\mathbb{Z}}
\newcommand{\argmax}{\operatorname*{\mathrm{arg\,max}}}
\newcommand{\argmin}{\operatorname*{\mathrm{arg\,min}}}
\renewcommand{\implies}{\Rightarrow}
\newcommand{\2}{\{0, 1\}}
\newcommand{\1}{\mathds{1}}

% Calculus/Analysis
\renewcommand{\d}{\mathrm{d}}
\newcommand{\Diff}[2][]{\frac{\d#1}{\d#2}}
\newcommand{\Grad}{\nabla}
\newcommand{\Del}[2][]{\frac{\partial#1}{\partial#2}}

% Linear algebra
\newcommand{\tr}{\mathrm{tr}}
\newcommand{\lmin}{\lambda_{\min}}
\newcommand{\lmax}{\lambda_{\max}}

% Miscellaneous
\newcommand{\eps}{\epsilon}
\newcommand{\cA}{\mathcal{A}}
\newcommand{\cB}{\mathcal{B}}
\newcommand{\cC}{\mathcal{C}}
\newcommand{\cD}{\mathcal{D}}
\newcommand{\cE}{\mathcal{E}}
\newcommand{\cF}{\mathcal{F}}
\newcommand{\cG}{\mathcal{G}}
\newcommand{\cH}{\mathcal{H}}
\newcommand{\cI}{\mathcal{I}}
\newcommand{\cJ}{\mathcal{J}}
\newcommand{\cK}{\mathcal{K}}
\newcommand{\cL}{\mathcal{L}}
\newcommand{\cM}{\mathcal{M}}
\newcommand{\cN}{\mathcal{N}}
\newcommand{\cO}{\mathcal{O}}
\newcommand{\cP}{\mathcal{P}}
\newcommand{\cQ}{\mathcal{Q}}
\newcommand{\cR}{\mathcal{R}}
\newcommand{\cS}{\mathcal{S}}
\newcommand{\cT}{\mathcal{T}}
\newcommand{\cU}{\mathcal{U}}
\newcommand{\cV}{\mathcal{V}}
\newcommand{\cW}{\mathcal{W}}
\newcommand{\cX}{\mathcal{X}}
\newcommand{\cY}{\mathcal{Y}}
\newcommand{\cZ}{\mathcal{Z}}
\newcommand{\Chi}{\cX}
\newcommand{\sgn}{\mathrm{sgn}}

\newcommand{\new}[1]{{\color{red} #1}}

% Project specific commands
%

\newcounter{nLectures}
\newcounter{nTheorems}[nLectures]

\newtheorem{theorem}[nTheorems]{Theorem}
\newtheorem{corollary}[nTheorems]{Corollary}
\newtheorem{conjecture}[nTheorems]{Conjecture}
\newtheorem{lemma}[nTheorems]{Lemma}
\newtheorem{proposition}[nTheorems]{Proposition}
\newtheorem{protocol}[nTheorems]{Protocol}
\newtheorem{claim}[nTheorems]{Claim}
\newtheorem{fact}[nTheorems]{Fact}

\theoremstyle{definition}
\newtheorem{definition}[nTheorems]{Definition}
\newtheorem{problem}[nTheorems]{Problem}
\newtheorem{intuition}[nTheorems]{Intuition}
\newtheorem{idea}[nTheorems]{Idea}
\newtheorem{exercise}[nTheorems]{Exercise}
\newtheorem{remark}[nTheorems]{Remark}
\newtheorem{formula}[nTheorems]{Formula}
\newtheorem{example}[nTheorems]{Example}

\makeatletter
	\let\c@algorithm\c@nTheorems
\makeatother

\usepackage{xpatch}
\makeatletter
\AtBeginDocument{\xpatchcmd{\@thm}{\thm@headpunct{.}}{\thm@headpunct{}}{}{}}
\makeatother

\hbadness=10000
\vbadness=10000

\setlength{\oddsidemargin}{.25in}
\setlength{\evensidemargin}{.25in}
\setlength{\textwidth}{6in}
\setlength{\topmargin}{-0.4in}
\setlength{\textheight}{9in}

\newcommand{\lecture}[4]{
	\setcounter{nLectures}{#1}
	\renewcommand{\thenTheorems}{\arabic{nLectures}.\arabic{nTheorems}}
	\renewcommand{\thealgorithm}{\arabic{nLectures}.\arabic{nTheorems}}
	\noindent
	\begin{center}
	\framebox{
		\vbox{
		\hbox to 5.78in { {\bf MAT 115B Number Theory
		\hfill Winter 2025} }
		\hbox to 5.78in { {\it \hfill #4} }
		\vspace{4mm}
		\hbox to 5.78in { {\Large \hfill Lecture \arabic{nLectures} \hfill} }
		\vspace{2mm}
		\hbox to 5.78in { {\it Lecturer: Elena Fuchs \hfill Scribe: #3} }
		}
	}
	\end{center}
	\vspace*{4mm}
}

\begin{document}
\lecture{4}{}{Avery Li}{January 13, 2025}

\section{Lecture 3 Recap}

\begin{definition}
    For $n=p_1^{r_1}\dots p_k^{r_k}$, $\mu(n)$ is the M\"obius function, defined as
    \begin{align*}
        \mu(1)&=1,\\
        \mu(n)&=
        \begin{cases} 
            (-1)^k &\text{ if } \forall i, r_i=1\\
            0 &\text{ otherwise}
         \end{cases}
    \end{align*}
\end{definition}
Essentially, $\mu(n)=0$ if it is divisible by a square and $(-1)^k$ otherwise. We showed that $\mu(n)$ is multiplicative.

\begin{example}
    Now we ask, what is the value of $f(x)=\sum_{d\vert x, d\geq 1}\mu(d)$?
    We have that
    \begin{align*}
        f(15)&=\mu(1)+\mu(3)+\mu(5)+\mu(15)\\
        &=1-1-1+1=0,\\
        f(8)&=\mu(1)+\mu(2)+\mu(4)+\mu(8)\\
        &=1-1-1+1=0,\\
        f(12)&=\mu(1)+\mu(2)+\mu(3)+\mu(4)+\mu(6)+\mu(12)\\
        &=1-1-1-0+1+0=0,\\
        f(1)&=\mu(1)=1.\\
    \end{align*}
    Consider the function \[ \sum_{d\vert n, d\geq 1}\mu(d)\nu(\frac{n}{d}),\]
    then
    \begin{align*}
        F(5)&=\mu(1)\nu(5)+\mu(5)\nu(1)\\
        &=1\cdot 2+(-1)\cdot 1=1,\\
        F(8)&=\mu(1)\nu(8)+\mu(2)\nu(4)+\mu(4)\nu(2)+\mu(8)\nu(1)\\
        &=1\cdot 4-1\cdot 3+0+0=1.\\
    \end{align*}
\end{example}

\section{M\"obius Inversion}
\begin{remark}\label{rem:rem_4_3}
    $\{d\vert n\}=\{\frac{n}{d}\ \forall d\vert n\}$
\end{remark}
\begin{fact}
    $f(n)=\sum_{d\vert n, d\geq 1\mu(d)}=
        \begin{cases} 
            1 \text{ if } n=1\\
            0\text{ otherwise} 
         \end{cases}
    $
\end{fact}
\begin{theorem}[Moebius Inversion (MI)]
    Let $f, g$ be arithmetic functions. Then
    \[  f(n)=\sum_{d\vert n, d\geq 1}g(d) \Longleftrightarrow g(n)=\sum_{d\vert n, d\geq 1}\mu(d)f\Big(\frac{n}{d}\Big). \]
\end{theorem}
\begin{proof}
    $(\Longrightarrow)$ Assume $g(n)=\sum_{d\vert n, d\geq 1}f(d)$, then
    \begin{align*}
        \sum_{d\vert n}\mu(d)f\Big(\frac{n}{d}\Big)&=\sum_{d\vert n}\mu\Big(\frac{n}{d}\Big)f(d)& (\hyperref[rem:rem_4_3]{4.3})\\
        &=\sum_{d\vert n}\mu\Big(\frac{n}{d}\Big)\sum_{d'\vert d}g(d')\\
        &=\sum_{d'\vert d\vert n}g(d')\mu\Big(\frac{n}{d}\Big)\\
    \end{align*}
    Note that $d'\vert d\implies \frac{n}{d}\vert \frac{n}{d'}$, so we can rewrite the sum as
    \begin{align*}
        \sum_{d'\vert d\vert n}g(d')\mu\Big(\frac{n}{d}\Big)&=\sum_{d'\vert n}g(d')\sum_{m\vert \frac{n}{d'}}\mu(m)\\
    \end{align*}
    Finally, terms of the inner sum are all $0$ except when $d'=n$, so we have
    \begin{align*}
        \sum_{d'\vert n}g(d')\sum_{m\vert \frac{n}{d'}}\mu(m)&=g(n)\mu(1)=g(n).\\
    \end{align*}
    $(\Longleftarrow)$
    \textcolor{blue}{\textit{To be written.}}

\end{proof}


\end{document}